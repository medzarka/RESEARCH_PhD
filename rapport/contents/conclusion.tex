\chapter{Conclusions and Perspectives}
\label{conlusion}

	The general framework of this dissertation is video information retrieval. 
	The main tackled challenge is how to enable a user to easily access and interpret 
	video contents. Particularly, our thesis works are focused on a contextual 
	concept-based video indexing that represents a motivating solution to such a challenge. 
	The major inherent difficulty in this task is the semantic gap: what separates the signal
	representation from semantics (concept, contexts and relationships).

	In multimedia indexing and retrieval literature, the multimedia community primarily focused
	on concept model building through a supervised learning technique that analyzes the extracted 
	low-level features: representative features vectors are extracted from a set of representative 
	manually annotated data (commonly images), then used  for a supervised learning algorithm in order 
	to approach the semantic interpretation. Nevertheless, such an approach partially solved the semantic gap 
	problem: it is difficult to detect efficiently a variety of concepts, and still unable to detect 
	implicit objects (semantic concepts that do not figure in the content, or that \revAnglais{cannot} be easily 
	detected). On the one hand, such issues are induced by the large variety of semantics that could be handled, 
	on the other hand, the inability of these approaches taking into consideration semantic concept relationships.
	Thus, recent research works are focusing on using ontologies for multimedia retrieval in order
	to allow semantic interpretation and reasoning over extracted descriptions. However, much
	remains to be done in order to achieve less human aid ontology modeling approaches.

	\section{Summary of Contributions}

	Our thesis work proposed a generic and scalable fuzzy knowledge based framework to enhance 
	concept-based multimedia indexing. In this context, our contributions aim to improve the accuracy, 
	the scalability and the generalization capability of our semantic video indexing system: \textsc{RegimVid}. 
	These novelties are enumerated as follows: 
	(1) a new knowledge based model for multimedia indexing. 
	The objective is to develop a framework able to handle various information about a multimedia content, 
	then to operate with \revAnglais{this} information in order to infer new information/knowledge through a reasoning process. 
	Such a novel model has to define and highlight pertinent components for an efficient knowledge based indexing process.
	(2) an ontology based framework to handle fuzzy knowledge and reason with semantic interpretations 
	in order to enhance and enrich them. This objective aims to define a semantic structure to model 
	required knowledge, then to specify an automated ontology population from available annotated 
	image/video datasets, and finally to handle ontology content evolving to further improve 
	semantics capabilities through \revAnglais{analyzing} and revising inaccurate and irrelevant knowledge.
	(3) an approach for an automatic multimedia indexing by the use of an ontology-based semantic 
	hierarchy handled at both learning and annotation steps. 

	While recent works focused on the use of semantic hierarchies to improve concept detector accuracy, 
	this objective means the use of such hierarchies to reduce detector complexity and then, to handle 
	efficiently large-scale multimedia datasets.


%%%%%%%%%%%%%%%%%%%%%%%%%%%%%%%
	\section{Future Research Directions}
	Many approaches were proposed in this dissertation in order to deal with video indexing 
	issues. Indeed, the semantic gap problem remains an open problem and could not be solved in 
	the near future. Therefore, research works on video indexing are witnessing an incremental
	improvement, and many supplementary efforts are still needed.

	In the following, we discuss some potential future directions that could be explored further over 
	the described achievements throughout this dissertation.

\begin{description}
	\item[Fuzzy Similarities]   Defining ontologies content was based on computing similarities 
		between concepts granted by a large annotated images dataset. Although we used the 
		statistical \emph{cosine} similarity function, more advanced fuzzy similarity 
		functions \citep{Baccour2013,Baccour2014} could be addressed in order to handle real 
		fuzzy knowledge to be inserted in proposed ontologies.
% 	\item[]    The ontology evolving approach proposed for our fuzzy ontology-based 
% 		system is quite preliminary. However, ontology evolving evaluation showed promising 
% 		results and proved that such an ontology revision should lead to a more pertinent 
% 		knowledge, then a better semantic interpretation enhancement. 
% 		Thus, we consider that extensive research works on 
% 		the ontology evolving part of our proposed approach should be more tackled.
	\item[Video genre]   We proposed a \revAnglais{knowledge} structure that enhances the accuracy of a semantic \revAnglais{interpretation}.
		Nevertheless, when analyzing a video annotated dataset, all videos are handled as if they were addressing the same subject. 
		Furthermore, the videos can be classified according to their genres. Thus, the relationships between 
		concepts/contexts can differ from one genre to another, and \revAnglais{subsequently}, from one video to another.
		In literature, the video classification is well-addressed 
		(\citep{Wu2012, Huang2012, Chattopadhyay2013, Muneesawang2014} to cite a few). 
		We are convinced that addressing the video genre as a semantic information can 
		further refine the extracted fuzzy relationships, and consequently, the video semantic 
		interpretation enhancement. Thus, we consider that it could be an interesting task that 
		should be followed for a future work. 
	\item[Big data video analysis ] 
		\revKarray{Many research works considers deep neural networks as a powerful
		 framework for big data video repositories
		 \citep{Girshick2014,Jiang2015,Sainath2015,Tong2015}. 
		 In \citep{Wu2015,Druzhkov2016}, a comprehensive survey on deep learning
		 and neural networks based approaches for video analysis and indexing is discussed.
		 The effectiveness  of such approaches resides in exploring parallel processing 
		 units (such as \textsc{Cuda} based machines \citep{Osipyan2015}) to run the 
		 neural networks pipelines. However, and as discussed in section 3.4, we are 
		 focusing more in our thesis work on a knowledge-based approach rather than 
		 a parallel one. }
		 
		 \revKarray{As a third contribution (discussed in chapter 6), we showed that a knowledge-based 
		 approach could reduce the computing cost of semantic concept detectors. 
		 But, we believe that this contribution should 
		 be extended to deep neural networks in order to achieve robust and real time 
		 multimedia content analysis.   }	
	
	\item[Domain Adaptation and Ontology evolving] 
		\revKarray{Ontology evolving is an important step in an ontology construction. 
		In fact, this step ensures not only the accuracy of the managed knowledge, but leads also to a continuous 
		adaptation of an ontology to a variety of application domains. The latter is based on the use of particular 
		approaches for constructing a discriminative model within a shift between observed data (training) and 
		analyzed one (test). Such a situation is widely discussed particularly when dealing with big data where 
		information is diverse and heterogeneous, and defined as \textit{domain adaptation} \citep{Pan2010}. 
		Many approaches were discussed in literature for the domain adaptation mainly by bridging the learned 
		(observed) and target (analyzed) domains by learning domain invariant features : the classifier learned 
		from observed domain can then be applied to the target domain \citep{Long2016}.}

		\revKarray{Recent studies are focusing particularly on deep neural networks to handle invariant features for domain 
		adaptation \citep{Donahue2014,Yosinski2014,Kumar2016}. Indeed, the domain adaptation is embedded in the pipeline 
		of deep feature learning in order to extract domain invariant representation 
		\citep{Tzeng2014,Long2015}.}

		\revKarray{In multimedia analysis, the domain adaptation leads to alleviate manual construction of 
		labeled data, and to enhance the ability to handle extended domains 
		\citep{Saenko2010,Gopalan2011,Gong2012,Duan2012a,Hoffman2014,Liu2016}. 
		Some recent works addressed the domain adaptation with more focus on deep neural networks 
		\citep{Ganin2015,Tzeng2015}.}

		\revKarray{Thus, the domain adaptation is an interesting research direction that should be tackled when 
		dealing with ontology population and evolving. In the present dissertation, we  opted to 
		use classical approaches to extract and evolve the content of an ontology. Thereby, 
		we believe that such a research direction will be highly considered as a future work.}

\end{description}

%	 \section{Future Development Directions}
%
%	\revKarray{As discussed in the previous section, the research presented in this dissertation 
%	leaves many promising avenues open for future research. This section looks at some 
%	of the future directions in the development of the multimedia retrieval systems.}
%
%	\revKarray{As hand-held and mobile devices are becoming common, the proposed framework to enhance
%	a semantic interpretation and to alleviate the complexity of semantic concept detection could 
%	be integrated within these devices. In fact, the use of semantic hierarchies to improve concept
%	detector accuracy could not only to handle efficiently large-scale multimedia contents, 
%	but also to enable efficient visual indexing with limited power and memory devices. 
%	Then, the semantic enhancement could enrich and improve incomplete semantic interpretation
%	using the ontology's deduction capabilities.}
%
%	\revKarray{For such a purpose, the proposed framework needs to be optimized so that it can be processed
%	with less power and memory capabilities.}

	


