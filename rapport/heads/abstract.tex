\begin{abstract}

Our thesis work deals with the video indexing based on semantic interpretation (an abstraction of objects or events that figure in a content), more particularly, the semantic indexing enhancement. Various approaches for semantic multimedia content analysis have been proposed addressing the discovery of features ranging from low-level features (color, histograms, sound \revAnglais{frequency}, motions, \dots{}) to high-level ones (semantic objects and concepts). However, these earlier approaches failed to reduce the semantic gap and were not able to deliver an accurate semantic interpretation. Under such a context, exploring further semantics within a multimedia content to improve semantic interpretation capabilities, is a major and a prerequisite challenge.

Towards exploring further semantic information within a multimedia content (other than low-level and semantic concepts one), valuable information (mainly concepts interrelationships and contexts) could be gathered from a multimedia content in order to enhance semantic interpretation capabilities. \revAnglais{Motivated} by a kindred vision of human perception, yet targeting automated analysis of a multimedia content, the multimedia retrieval community addressed more attention to multimedia ontologies.

Aiming to contribute towards this direction, we focus on modeling an automated fuzzy context-based ontology framework for enhancing a video indexing accuracy and efficiency. Key dimensions of this inquiry constitute the main issues addressed by the use of ontologies for multimedia indexing, namely: (1) the knowledge management and evolution, (2) the ability to handle uncertain knowledge and to deal with fuzzy semantics, and (3) the scalability and the ability to process a growing multimedia content volume with a continuous request for a better machine semantic interpretation capacities.

What was accomplished in our study is a novel ontology management which is intended to a machine-driven knowledge database construction. Such a method could enable semantic improvements in large-scale multimedia content analysis and indexing. 

In order to illustrate the semantic enhancement of concept detection introduced by our proposed scalable and generic ontology-based framework, we have conducted different experiments within three multimedia evaluation campaigns: \textsc{TrecVid 2010} (within \emph{Semantic Indexing} Task), \textit{ImageClef 2012} (within \emph{Photo Annotation and Retrieval} Task), and \textit{ImageClef 2015} (within \emph{Scalable Concept Image Annotation } Task).

\par ~
\par ~
\par ~
\par ~
	\textbf{\textit{Keywords:}} Video Indexing, Semantic Interpretation, Fuzzy Ontology, Fuzzy Reasoning, Hierarchical Concept Detector.


\end{abstract}

