\chapter*{Author's Response \\ to the Reviewers' Comments}

% ======================================================= %
%\noindent Dear editor and reviewers,\\

	We would like to thank the reviewers for their interest in our work and their
	relevant and helpful comments that will greatly improve and velarize the thesis manuscript. 
	Thus, we tried to do our best to respond to the raised points. 

	The reviewers have brought up good comments and the opportunity 
	to clarify our research objectives and results. In this document, we have
	checked all the comments provided by the reviewers and have made necessary 
	changes according polo 2017to their recommendations.

	For clarity, the comments are applied in the new thesis manuscript version 
	with \colorbox{babyblue}{\textsf{Blue}} color for the reviewer Professor Fakhri Karray, 
	with \colorbox{brightgreen}{\textsf{Green}} color for the reviewer Professor Sami Faiz, 
	and finally with \colorbox{yellow}{\textsf{Yellow}} color for typos and editorial mistakes.

	\vfill
	Yours sincerely, Mohamed ZARKA.


	\newpage

% ======================================================= %
\section*{Answers to reviewer: Professor Fakhri Karray}

\question{There	are several typos and editorial	mistakes and strongly recommend the candidate to carefully proofread the thesis.} 

\answer We apologize for the typos and the editorial misyakes. They were considered and corrected.


\question{The candidate made very good literature review and has highlighted recent work in the field. 
	Possibly more would have been proposed especially recent approaches highlighted in the literature dealing
	with big data based video clip repositories indexing and search using on-line learning and domain adaptation. 
	This is a very promising research direction and would wish to have the candidate mention 
	it as future potential direction.}

\answer ...


\question{The examiner would have wished to see more powerful performance metric such as recall performance 
	and computation requirement (algorithmic time performance).}

\answer ...

\question{Moreover, it is very important when dealing 
	with large data set to apply distributed/cloud computing environment to handle real-time classification
	and detection of new concepts/relationship. This requires a certain type of the algorithm structure, 
	not mentioned in details in this work. These are major issue on their own, and it is not expected 
	the candidate to deal with all of them here. He	simply needs to highlight them though in the final draft.}

\answer ...

\question{Hybrid approaches using existing solid techniques for concept detection coupled with machine learning
	algorithms to deal with concept relationship (context) is being proven very powerful. 
	Candidate should mention these approaches in his references.}

\answer ...


\question{I have some issues with the publication record (in terms of quantity and quality) 
	and strongly suggest that the candidate should really work hard on publishing his work in 
	well-known venues pertinent to the field.}

\answer We thank you for pointing this out. 

	In terms of publication quality, we published our second contribution within a well known journal 
	(Springer MTAP with impact factor of $1.331$). Actually, I am drafting a second journal paper that 
	deals with our third contribution in order to valorize the obtained results. 
	Yet, I target the following two journals for the submission:
	\textit{ACM Multimedia Computing, Communications, and Applications} (Impact Factor of $2.465$),
	or \textit{Springer International Journal of Computer Vision} (Impact Factor of $4.270$).

	In terms of publication quantity, I would like to notice that handling large amount of data was a 
	real hard task with the use of very classical computing machines. I will try to do my best to 
	increase my publication rate. Thank you again for this comment.

	As for industrial development, we prepared a patent  INNORPI (National Institute of 
	Standardization and Industrial Property). This patent is entitled \textit{“Dispositif d'Enrichissement 
	Sémantique en utilisant des ONTOlogies (DESONTO) pour l'amélioration de l'indexation des 
	contenus multimédias”}, and it details the semantic enhancement for multimedia retrieval systems 
	through the use of fuzzy ontologies (as described in our second contribution C2). 
	This patent is currently under review. 
	Furthermore, we are preparing a second patent about the scalability aspect. 
	This patent is entitled \textit{“Annotation Sémantique Rapide des Images en utilisant des Ontologies 
	(ASRIO) pour l'indexation des contenus visuels.”}, 
	and it proposes an alleviated computing task for semantic concept detection within multimedia contents. 
	This patent is being drafted.

	In order to highlight such a promising industrial development, we added a new section entitled 
	“\textit{7.3 Future Development Directions}” in the chapter “\textit{Conclusions and Perspectives}”.

% ==================================================================================================== %
% ==================================================================================================== %        
% ==================================================================================================== %        
% ==================================================================================================== %        
% ==================================================================================================== %        
% ==================================================================================================== %        
% ==================================================================================================== % 

\setcounter{commentNum}{0}
\section*{Answers to Reviewer: Professor Sami Faiz}

\question{Nous aurions juste aimé avoir une discussion au cas où les connaissances pouvant être elles mêmes floues.}


